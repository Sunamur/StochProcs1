\documentclass{article}

\usepackage{xecyr}
\setmainfont[Mapping=tex-text,Ligatures=TeX]{CMU Serif}

\title{Отчёт по проектной работе}
\author{Гармаева Д., Шерстюк Л., Калмыкова Н., Пеньков А.}
\date{}

\begin{document}

\maketitle

\section{Вводные}

Были даны уровни российских и американских процентных ставок на 01.01.2019, а также значения $\textit{mean reversion level}$ (среднее значение, к которому возвращается наш риск-фактор),$\alpha$ (скорость возвращения к среднему) и $\sigma$ (волатильность, или стандартное отклонение прироста процентной ставки) для модели Халла-Уайта. С помощью формулы 

$$F = S0*(1+rf)/(1+rd)$$ мы также могли получить обменный курс

\section{Задача}

\begin{enumerate}
    \item Оценить матрицу корреляций для наших вводных процентных ставок и обменного курса
    \item С помощью разложения Холецкого сгенерировать 3 соответствующих гауссовских вектора риск-факторов с заданным распределением
    \item Оценить максимальный скачок для риск-факторов
    \item При симуляциях оценить разность 95$\%$ квантилей и получить некий уровень точности, и оценить его экспертно
\end{enumerate}

\section{Шаги (за гиги)}

\begin{enumerate}
    \item Понимание теоретической базы задачи
    \begin{enumerate}
        \item Современные финансовые рынки предлагают множество финансовых инструментов, которые при правильной оценке и использовании могут помочь минимизировать уровень риска. Для оценки финансовых инструментов широко используются модели, одной из самых популярных на данный момент является модель Блэка-Шоулза. Однако модель Блэка-Шоулза обладает рядом ограничений – она работает в предположении постоянной волатильности и уровня процентных ставок.
        \item Для более точной оценки флуктуаций процентных ставок и процентных ставок можно применить модель временной структуры процентных ставок Халла-Уайта (которая является расширенной версией модели Васичека), где параметры α(t) и σ(t) зависят от времени:
            
            $$dr = [\theta(t) – \alpha(t)*r]*dt + \sigma(t)*dW,$$
            
            где $\theta (t) = \frac{df(t,T)}{dt} + \alpha f(t,T) + \frac{(1- e^{-2\alpha t}) \sigma^2}{2\alpha}$ обеспечивает точное соответствие модели и наблюдаемой на рынке структуры процентных ставок,
            
            $f(t,T)$ - форвардная ставка процента.
    \end{enumerate}
        \item Использование наших данных для построения модели:
            \begin{enumerate}
                \item Проводим интерполяцию данных рыночных процентных ставок на каждые две недели
                \item Получив на предыдущем шаге точные процентные ставки на годовом временном горизонте с шагом в две недели, и, соответственно, обменный курс, мы можем получить корреляционную матрицу наших риск-факторов 
                \item Для прогнозирования наших риск факторов на следующий год, мы используем совместную генерацию процентных ставок и обменного курса с помощью разложения Холецкого:
                \begin{itemize}
                    \item Исходные данные: положительно определённая симметрическая  матрица \textit{A} (элементы $a_{ij}$).
                    \item Вычисляемые данные: нижняя треугольная матрица \textit{L} (элементы $l_{ij}$).
                    \item Формулы метода:
                        $$l_{11} = \sqrt{a_{11}},$$
                        $$l_{j1} = \frac{a_{j1}}{l_{11}}, j \in [2,n]$$
                        $$l_{ii} = \sqrt{a_{ii} - \sum^{i-1}_{p=1}{l^2_{ip}}}, i \in [2,n]$$
                        $$l_{ji} = \frac{\big( a_{ji} - \sum^{i-1}_{p=1}{l_{ip} l_{jp}} \big)}{l_{ii}}, i \in [2,n-1], j \in [i+1,n]$$
                \end{itemize}
                \item Для каждого фактора используем предыдущие значения r, заданные значения $\alpha$ и полученные при разложении Холецкого $\sigma$.
                \end{enumerate}
        \item Оценка вероятности шока на рынке
        \begin{enumerate}
                \item Для данного шага нам надо посчитать 95-процентный квантиль на каждой двухнедельной итерации симуляционного блока и среди этих квантилей выбрать максимум
        \end{enumerate}
\end{enumerate}



\end{document}
